%% BioMed_Central_Tex_Template_v1.05
%%                                      %
%  bmc_article.tex            ver: 1.05 %
%                                       %


%%%%%%%%%%%%%%%%%%%%%%%%%%%%%%%%%%%%%%%%%
%%                                     %%
%%  LaTeX template for BioMed Central  %%
%%     journal article submissions     %%
%%                                     %%
%%         <27 January 2006>           %%
%%                                     %%
%%                                     %%
%% Uses:                               %%
%% cite.sty, url.sty, bmc_article.cls  %%
%% ifthen.sty. multicol.sty		       %%
%%									   %%
%%                                     %%
%%%%%%%%%%%%%%%%%%%%%%%%%%%%%%%%%%%%%%%%%


%%%%%%%%%%%%%%%%%%%%%%%%%%%%%%%%%%%%%%%%%%%%%%%%%%%%%%%%%%%%%%%%%%%%%
%%                                                                 %%	
%% For instructions on how to fill out this Tex template           %%
%% document please refer to Readme.pdf and the instructions for    %%
%% authors page on the biomed central website                      %%
%% http://www.biomedcentral.com/info/authors/                      %%
%%                                                                 %%
%% Please do not use \input{...} to include other tex files.       %%
%% Submit your LaTeX manuscript as one .tex document.              %%
%%                                                                 %%
%% All additional figures and files should be attached             %%
%% separately and not embedded in the \TeX\ document itself.       %%
%%                                                                 %%
%% BioMed Central currently use the MikTex distribution of         %%
%% TeX for Windows) of TeX and LaTeX.  This is available from      %%
%% http://www.miktex.org                                           %%
%%                                                                 %%
%%%%%%%%%%%%%%%%%%%%%%%%%%%%%%%%%%%%%%%%%%%%%%%%%%%%%%%%%%%%%%%%%%%%%


\NeedsTeXFormat{LaTeX2e}[1995/12/01]
\documentclass[10pt]{bmc_article}    



% Load packages
\usepackage{cite} % Make references as [1-4], not [1,2,3,4]
\usepackage{url}  % Formatting web addresses  
\usepackage{ifthen}  % Conditional 
\usepackage{multicol}   %Columns
\usepackage[utf8]{inputenc} %unicode support
%\usepackage[applemac]{inputenc} %applemac support if unicode package fails
%\usepackage[latin1]{inputenc} %UNIX support if unicode package fails
\urlstyle{rm}
 
 
%%%%%%%%%%%%%%%%%%%%%%%%%%%%%%%%%%%%%%%%%%%%%%%%%	
%%                                             %%
%%  If you wish to display your graphics for   %%
%%  your own use using includegraphic or       %%
%%  includegraphics, then comment out the      %%
%%  following two lines of code.               %%   
%%  NB: These line *must* be included when     %%
%%  submitting to BMC.                         %% 
%%  All figure files must be submitted as      %%
%%  separate graphics through the BMC          %%
%%  submission process, not included in the    %% 
%%  submitted article.                         %% 
%%                                             %%
%%%%%%%%%%%%%%%%%%%%%%%%%%%%%%%%%%%%%%%%%%%%%%%%%                     


\def\includegraphic{}
\def\includegraphics{}



\setlength{\topmargin}{0.0cm}
\setlength{\textheight}{21.5cm}
\setlength{\oddsidemargin}{0cm} 
\setlength{\textwidth}{16.5cm}
\setlength{\columnsep}{0.6cm}

\newboolean{publ}

%%%%%%%%%%%%%%%%%%%%%%%%%%%%%%%%%%%%%%%%%%%%%%%%%%
%%                                              %%
%% You may change the following style settings  %%
%% Should you wish to format your article       %%
%% in a publication style for printing out and  %%
%% sharing with colleagues, but ensure that     %%
%% before submitting to BMC that the style is   %%
%% returned to the Review style setting.        %%
%%                                              %%
%%%%%%%%%%%%%%%%%%%%%%%%%%%%%%%%%%%%%%%%%%%%%%%%%%
 

%Review style settings
\newenvironment{bmcformat}{\begin{raggedright}\baselineskip20pt\sloppy\setboolean{publ}{false}}{\end{raggedright}\baselineskip20pt\sloppy}

%Publication style settings
%\newenvironment{bmcformat}{\fussy\setboolean{publ}{true}}{\fussy}



% Begin ...
\begin{document}
\begin{bmcformat}


%%%%%%%%%%%%%%%%%%%%%%%%%%%%%%%%%%%%%%%%%%%%%%
%%                                          %%
%% Enter the title of your article here     %%
%%                                          %%
%%%%%%%%%%%%%%%%%%%%%%%%%%%%%%%%%%%%%%%%%%%%%%

\title{Parallelizing Cheminformatics with Map/Reduce and Hadoop}
 
%%%%%%%%%%%%%%%%%%%%%%%%%%%%%%%%%%%%%%%%%%%%%%
%%                                          %%
%% Enter the authors here                   %%
%%                                          %%
%% Ensure \and is entered between all but   %%
%% the last two authors. This will be       %%
%% replaced by a comma in the final article %%
%%                                          %%
%% Ensure there are no trailing spaces at   %% 
%% the ends of the lines                    %%     	
%%                                          %%
%%%%%%%%%%%%%%%%%%%%%%%%%%%%%%%%%%%%%%%%%%%%%%


\author{Rajarshi Guha\correspondingauthor$^{1}$%
       \email{Rajarshi Guha\correspondingauthor - guhar@mail.nih.gov}%
      \and
         Jane E Doe\correspondingauthor$^2$%
         \email{Jane E Doe\correspondingauthor - jane.e.doe@cambridge.co.uk}
     }
      

%%%%%%%%%%%%%%%%%%%%%%%%%%%%%%%%%%%%%%%%%%%%%%
%%                                          %%
%% Enter the authors' addresses here        %%
%%                                          %%
%%%%%%%%%%%%%%%%%%%%%%%%%%%%%%%%%%%%%%%%%%%%%%

\address{%
    \iid(1)NIH Chemical Genomics Center, 9800 Medical Center Drive, Rockville, %
    MD 20852
}%

\maketitle

%%%%%%%%%%%%%%%%%%%%%%%%%%%%%%%%%%%%%%%%%%%%%%
%%                                          %%
%% The Abstract begins here                 %%
%%                                          %%
%% The Section headings here are those for  %%
%% a Research article submitted to a        %%
%% BMC-Series journal.                      %%  
%%                                          %%
%% If your article is not of this type,     %%
%% then refer to the Instructions for       %%
%% authors on http://www.biomedcentral.com  %%
%% and change the section headings          %%
%% accordingly.                             %%   
%%                                          %%
%%%%%%%%%%%%%%%%%%%%%%%%%%%%%%%%%%%%%%%%%%%%%%


\begin{abstract}
  \paragraph*{Background:} Many techniques and software frameworks
  have been designed to support parallelization. The map/reduce
  paradigm is one such technique that is based on the application of a
  user defined function to a series of objects, the results of which
  are then collected and processed. Recently, software frameworks have
  been developed that allow one to employ this technique on large
  compute clusters. In this paper we discuss how various
  cheminformatics applications can be rewritten to make use of the
  map/reduce paradigm. Specifically, we focus on the use of the Hadoop
  framework to implement these applications. We also present
  performance benchmarks comparing Hadoop applications to their serial
  counterparts. Finally, we provide a brief description of how such
  applications can benefit from elastic computing providers, viz.,
  Amazon EC2.
      
        \paragraph*{Results:} Text for this section of the abstract \ldots

        \paragraph*{Conclusions:} Our results indicate that many
        cheminformatics applications can be easily converted to a
        map/reduce style and using the Hadoop framework can be
        seamlessly deployed on to large Hadoop clusters, such as as
        those available from Amazon. Performance benchmarks indicate
        that such deployments need relatively large datasets to show
        significant improvements. At the same time, the use of
        commercial elastic computing providers such as Amazon, does
        have downsides primarily due to the bandwidth involved in
        loading datasets and retreiving results.
\end{abstract}



\ifthenelse{\boolean{publ}}{\begin{multicols}{2}}{}




%%%%%%%%%%%%%%%%%%%%%%%%%%%%%%%%%%%%%%%%%%%%%%
%%                                          %%
%% The Main Body begins here                %%
%%                                          %%
%% The Section headings here are those for  %%
%% a Research article submitted to a        %%
%% BMC-Series journal.                      %%  
%%                                          %%
%% If your article is not of this type,     %%
%% then refer to the instructions for       %%
%% authors on:                              %%
%% http://www.biomedcentral.com/info/authors%%
%% and change the section headings          %%
%% accordingly.                             %% 
%%                                          %%
%% See the Results and Discussion section   %%
%% for details on how to create sub-sections%%
%%                                          %%
%% use \cite{...} to cite references        %%
%%  \cite{koon} and                         %%
%%  \cite{oreg,khar,zvai,xjon,schn,pond}    %%
%%  \nocite{smith,marg,hunn,advi,koha,mouse}%%
%%                                          %%
%%%%%%%%%%%%%%%%%%%%%%%%%%%%%%%%%%%%%%%%%%%%%%




%%%%%%%%%%%%%%%%
%% Background %%
%%
\section*{Background}

The map / reduce paradigm has been discussed in various forms since
XXX [REFS], and describes an approach to parallel
processing. Fundamentally, the paradigm consists of two steps. In the
\textit{map} phase, a task is broken down into small chunks, each of
which may be processed on different nodes of a cluster (or different
cores on multi-core CPU). The output of each of the tasks is then
aggregated in the \textit{reduce} phase. A more formal definition is
given below.

Given that map / reduce can be considered a design pattern [REF],
there are many implementations of this in a variety of languages. For
example, the map phase is easily carried out in languages such as
Python (the \texttt{map} command), R (the \texttt{apply} or
\texttt{lapply} commands) and Lisp (the \texttt{map} command). Thus
while many languages support the map/reduce paradigm, the real value
is the ability to use this approach to process massive datasets over
large compute clusters. This scenario ws described in by Ghemawat and
has spurred the development of a number of map/reduce
\textit{frameworks}, which allow one to take advantage of large
clusters. Probably the most well known such framework is Hadoop, first
developed by Yahoo Inc. and subsequently open sourced. The framework
allows one to process arbitrarily large datasets, over very large
clusters. For example, Yahoo has reported the analysis of
multi-terabyte web log files on clusters with 4000 nodes [REF].

Many applications of the map/reduce approach have focused on
\textit{trivially parallelizable} problems. In these cases, the
computational processing step is relatively simple, but must be
applied a large number of objects (which may be individual files or
even individual lines in a single, very large file). Examples of this
include log file analysis [REF], image process [REF] and document
analysis [REF]. On the other hand there are a number of cases, where
traditional serial algorithms can be converted in to a map/reduce
framework In such cases, the dataset itself may not be particularly
large, but the computational processing can be sped up by being
distributed over a cluster. Examples of such appplications include
linguistic analysis [REF], outlier detection [REF]. In addition, there
have been reports of a variety of machine learning algorithms being
converted to the map/reduce framework including Bayesian networks
[REF], discriminant classification [REF] and XXX.

While there has been active development of map/reduce fragments in a
variety of fields, such developments for life science applications
have not been as widespread. Most of the current scientific
applications have focused on bioinformatics methodologies. A recent
example is the case of Bowtie [REF], which is an implementation of
genome assembly algorithms in the map/reduce framework.

In contrast, there have been reports of cheminformatics applications
making use of the map/reduce pardigm, either at the dataset or
algorithmic levels. One of the major reasons for this is that most
datasets in cheminformatics are not very large (at least compared to
those generated in other fields such as genome sequencing). On an
algorithmic level, certain cheminformatics methods (such as thosed
based on machine learning) can make use of map/reduce
frameworks. Whether it is useful to do so brings us back to the issue
of dataset size.

The rest of this paper describes the use of the Hadoop map/reduce
framework for cheminformatics problems. Specifically we focus on the
use of Hadoop version 0.18. While more recent versions are available,
this is the version that is support by Amazon, thus allowing us to
test our applications on the Amazon cloud, without modifications to
the source code. Section XXX describes how one designs a Hadoop
program to perform such computations. We specifically highlight
input/output issues that arise when one moves from line oriented
formats (i.e., SMILES files) to multi-line record formats (such as SD
files). We then describes examples of map/reduce applications. We
consider data-parallel problems as well as algorithm-parallel
applciations. Section XXX provides some benchmarks comparing Hadoop
programs to their serial counterparts. We also provide some initial
benchmarks for programs running on Amazon Elastic Compute
servers. Finally, Section XXX discusses some of the drawbacks of the
map/reduce paradigm and use of Amazon services as well as further
applications in cheminformatics.


%%%%%%%%%%%%%%%%%%%%%%%%%%%%
%% Results and Discussion %%
%%
\section*{Results and Discussion}

In the following subsections we discuss a number of cheminformatics
applications using the Hadoop framework.

\subsection*{Atom counting}
\label{sec:atom-counting}

We first consider the cheminformatics equivalent of the canonical word counting Hadoop application. In the original example, the goal is to calculate the frequency of words in a very large text file. In the cheminformatics version, we are interested in counting the frequency of individual elements in a collection of SMILES strings. The program is provided in Listing ZZZ. The key features are the mapper and reducer classes. In the former, the *map* method receives a single line from the SMILES file, along with an arbitrary key. The line is parsed as SMILES string giving us an object representing the molecules.  We then loop over the atoms in the molecule and for each atom we write out the symbol along with a constant integer (viz., 1). This represents the output of the mapper - a series of key, value pairs, in which the key is the atomic symbol and its value simply indicates that the symbol occurred. One can see that by summing up the values for a given key, one would obtain the number of times that the element occurs in the collection of SMILES strings. Note that the output of the mapper is written to a special object of class Context (as opposed to STDOUT). This allows the Hadoop framwork to aggregate the key,value output of the mappers (which are likely running on different nodes) which can then be provided to the reducer class.

At one point, after sufficient results from the mappers have been obtained, the framework will instantiate the reducer class. In this example, the reducer class will receive a single key and the values associated with that key. For the atom counting case, a key could be the symbol "C" and the values would be the series {1,1,1,1,1}. In this application the reduce operation need only sum up the values and as with the mapper, writes out the key and it's reduced value to the Context object. Hadoop can then collect the output of the reducers (which will in general be running on multiple nodes) and generate the final output of the program.

it is important to note that the output type of the mapper should match the input type of the reducer. In this example, the output of the mapper is a series of String,Integer pairs (the variable "one" is an instance of IntWritable). As a result, the reducer will expect that the values corresponding to a given key to be a series of Integers. Thus the second argument of the reduce method in IntSumReducer is Iterable<IntWritable>. This is also indicated in the signature of the mapper and reducer classes. 

\subsection*{Benchmarks}
\label{sec:benchmarks}

To run the applications described here on ones local machine, one must set up a Hadoop File System (HFS). Into this filesystem, one loads the input data files and on completion of the run, one retrieves the output files. For Hadoop version 0.18 one can initialize the HFS as follows:

foo
bar

With the filesystem initialized, one copies a data file from the local disk into the HFS by:

foo
bar

At this point we can run the program. We compile the program into a single jar file, contaning all the dependencies. With this jar file (all programs described here can be obtained in JAR form from XXX) one then runs the program as

foo
bar

Progress can usually be monitored via a web browser at XXX, allowing one to view the progress of the mappers and reducers. Obviously, on a single machine this is not particularly informative, but is useful for debugging purposes. One benefit of using the Hadoop provided configuration classes is that one can switch to serial execution from the command line. In this mode, the input file is still chunked, but each chunk is processed serially. This allows us to easily generate a number of benchmarks.

\subsection*{Discussion}
\label{sec:discussion}

* Why not use a job scheduling system - these address the map phase,
but would require extra work to perform the reduce phase.

* The nature of the file *chunking* step suggests the use of very
large input files. For applications based on SMILES input, it is rare
to have multi-gigabyte input files (the whole of PubChem in SMILES
format is XXX GB). In such cases, one can reduce the default chunk
size, allowing the use of multiple mapper nodes. However, the amount
of data in a given chunk becomes smaller and the overhead of the data
distrbution step may overshadow the actual computation time. This
issue becomes unimportant when dealing with large SD files.

* map reduce is bad -
\url{http://developers.slashdot.org/story/08/01/18/1813248/MapReduce-mdash-a-Major-Step-Backwards}


    

%%%%%%%%%%%%%%%%%%%%%%
\section*{Conclusions}
While most cheminformatics applications do not yet deal with datasets
as large those generated in other fields (such as genome sequencing),
we show that the use of map/reduce allows such applications to scale
to massive datasets when they arrive.


  
%%%%%%%%%%%%%%%%%%
\section*{Methods}
Hadoop is a general purpose Java framework to support map/reduce style
programs. While Hadoop programs can be run on a single machine with a
single CPU, it's primary goal is to support such applications on large
clusters. Thus, one can implement a Hadoop program and test it on ones
local machine and then deploy it as is to a large cluster for
production runs. It should also be noted that Hadoop programs do not
necessarily have to be Java programs. It is posible to use programs
written in arbitrary languages with Hadoop, as long they can read from
standard input and write to standard output. In addition to the Hadoop
framework, we require the use of a cheminformatics library to actually
perform the domain tasks. We employ the Chemistry Development Kit
(CDK), an Open Source Java cheminformatics toolkit [REFS].

\subsection*{Data Flow in Hadoop Programs}
\label{sec:data-flow-hadoop}
To improve I/O efficiency, most Hadoop programs prefer to work with a
few large files, rather a large number of small files. We assume that
a single large file is to be processed. When a Hadoop program is
started, it will *chunk* the file, sending each chunk to an individual
node. For text files, a chunk is defined in terms of bytes - by
default, each chunk is 64MB, though this can be changed at run
time. In general, if the processing time for an individual record is
small, larger chunks will be preferred.  Note that since chunks are
defined by size, there is no guarantee that a chunk begins at the
start of a record or ends at the end of record. In the case of data
files where each line is a record, Hadoop will automatically provide
complete lines to the mapper classes. For multi-line records, the user
must provide an appropriate reader. Once a chunk has been created and
sent to a node, it is then processed by the mapper class, which will
receive one record at a time. Each node will process the records in
its chunk and emit a new set of records in the form <key, value>. Once
the first set of records are available from the mappers, the reducer
classes will aggregate the key, value pairs such that all pairs with
the same key are collected and then processed. In simple Hadoop
programs, the output of the reducer classes is the output of the
program. However, there is no restriction on the number of map and
reduce phases, so that the output from the first reduce phase could be
fed back into a second map phase. It is also possible to avoid the
reduce phase in some cases (see Section XXX). While always having a
reduce phase does not affect the final results, being able to avoid a
reduce phase can lead to more efficient Hadoop programs.

\subsection*{Mappers and Reducers}
\label{sec:mappers-reducers-1}

The key to developing a Hadoop program is to implement classes for the
map and reduce phases.  A schematic outline of these two classes are
shown in Listings XXX and YYY. When the mapper class is instantiated
and called, it receives a single record as a String object.


\subsection*{Input/Output}
\label{sec:inputoutput}

By default, Hadoop allows one to write map/reduce programs that deal
with single-line oriented data files. That is each line represents a
single record. In cheminformatics, SMILES files are an example of such
records. Thus if the input is a SMILES file, one need not worry about
file processing and can simply implement the map and reduce classes.

However many applications require the use of multi-line records such
as SD files, CML files and so on. To support such files, so that the
mappers are able to access individual records, one must implement
custom readers. We have provided a a class to support the reading of
SD files in a Hadoop program.
    
%%%%%%%%%%%%%%%%%%%%%%%%%%%%%%%%
\section*{Authors contributions}
RG devised the applications, implemented the Hadoop programs and wrote the paper.

    

%%%%%%%%%%%%%%%%%%%%%%%%%%%
\section*{Acknowledgements}
  \ifthenelse{\boolean{publ}}{\small}{}
  Text for this section \ldots


 
%%%%%%%%%%%%%%%%%%%%%%%%%%%%%%%%%%%%%%%%%%%%%%%%%%%%%%%%%%%%%
%%                  The Bibliography                       %%
%%                                                         %%              
%%  Bmc_article.bst  will be used to                       %%
%%  create a .BBL file for submission, which includes      %%
%%  XML structured for BMC.                                %%
%%                                                         %%
%%                                                         %%
%%  Note that the displayed Bibliography will not          %% 
%%  necessarily be rendered by Latex exactly as specified  %%
%%  in the online Instructions for Authors.                %% 
%%                                                         %%
%%%%%%%%%%%%%%%%%%%%%%%%%%%%%%%%%%%%%%%%%%%%%%%%%%%%%%%%%%%%%


{\ifthenelse{\boolean{publ}}{\footnotesize}{\small}
 \bibliographystyle{bmc_article}  % Style BST file
  \bibliography{bmc_article} }     % Bibliography file (usually '*.bib' ) 

%%%%%%%%%%%

\ifthenelse{\boolean{publ}}{\end{multicols}}{}

%%%%%%%%%%%%%%%%%%%%%%%%%%%%%%%%%%%
%%                               %%
%% Figures                       %%
%%                               %%
%% NB: this is for captions and  %%
%% Titles. All graphics must be  %%
%% submitted separately and NOT  %%
%% included in the Tex document  %%
%%                               %%
%%%%%%%%%%%%%%%%%%%%%%%%%%%%%%%%%%%

%%
%% Do not use \listoffigures as most will included as separate files

\section*{Figures}
  \subsection*{Figure 1 - Sample figure title}
      A short description of the figure content
      should go here.

  \subsection*{Figure 2 - Sample figure title}
      Figure legend text.



%%%%%%%%%%%%%%%%%%%%%%%%%%%%%%%%%%%
%%                               %%
%% Tables                        %%
%%                               %%
%%%%%%%%%%%%%%%%%%%%%%%%%%%%%%%%%%%

%% Use of \listoftables is discouraged.
%%
\section*{Tables}
  \subsection*{Table 1 - Sample table title}
    Here is an example of a \emph{small} table in \LaTeX\ using  
    \verb|\tabular{...}|. This is where the description of the table 
    should go. \par \mbox{}
    \par
    \mbox{
      \begin{tabular}{|c|c|c|}
        \hline \multicolumn{3}{|c|}{My Table}\\ \hline
        A1 & B2  & C3 \\ \hline
        A2 & ... & .. \\ \hline
        A3 & ..  & .  \\ \hline
      \end{tabular}
      }
  \subsection*{Table 2 - Sample table title}
    Large tables are attached as separate files but should
    still be described here.



%%%%%%%%%%%%%%%%%%%%%%%%%%%%%%%%%%%
%%                               %%
%% Additional Files              %%
%%                               %%
%%%%%%%%%%%%%%%%%%%%%%%%%%%%%%%%%%%

\section*{Additional Files}
  \subsection*{Additional file 1 --- Sample additional file title}
    Additional file descriptions text (including details of how to
    view the file, if it is in a non-standard format or the file extension).  This might
    refer to a multi-page table or a figure.

  \subsection*{Additional file 2 --- Sample additional file title}
    Additional file descriptions text.


\end{bmcformat}
\end{document}







